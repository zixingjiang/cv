%!TeX program = xelatex
% Zixing Jiang's Curriculum Vitae
% Email: zxjiang@surgery.cuhk.edu.hk
% Web: https://zixingjiang.com/
% Repo: https://github.com/zixingjiang/cv

\documentclass[utf8, 11pt,letterpaper]{report}
\usepackage[fontset=fandol]{ctex}
\usepackage[T1]{fontenc} % output T1 font encoding (8-bit) for accented characters as single glyph
\usepackage[strict,autostyle]{csquotes} % smart and nestable quote marks
\usepackage[USenglish]{babel} % regionalize hyphens, quote marks, etc automatically
\usepackage{microtype}% improve text appearance with kerning, etc
\usepackage{datetime} % enable formatting of date output
\usepackage{tabto}    % make nice tabbing
\usepackage{setspace} % custom line spacing (should load this before hyperref)
\usepackage{hyperref} % enable hyperlinks and pdf metadata
\usepackage{geometry} % manually set page margins
\usepackage{enumitem} % enumerate with [resume] option
\usepackage{titlesec} % allow custom section fonts
\usepackage[dvipsnames]{xcolor}
\usepackage[fixed]{fontawesome5}
\usepackage{textcase}

% what is your name?
\newcommand{\myname}{蒋子星}

% select default typefaces
\usepackage{crimson} % document's serif font
\usepackage{helvet}  % document's sans serif font

%\usepackage{ebgaramond} % document's serif typeface
%\usepackage{tgheros}    % document's sans serif typeface

% how far to tab for list items with left-aligned date: different fonts need different widths
\newcommand{\listtabwidth}{1.8cm}

% define font to use as document's title
\newcommand{\namefont}[1]{{\normalfont\bfseries\LARGE{#1}}}


% set section heading fonts and before/after spacing
\SetTracking{encoding=*, family=\sfdefault}{30} % increase sans serif headings tracking
\titleformat{\section}{\CJKfamily{zhyahei}\sffamily}{}{}{}{}
\titlespacing{\section}{0pt}{18pt plus 4pt minus 4pt}{6pt plus 2pt minus 2pt}

% set subsection heading fonts and before/after spacing
\titleformat{\subsection}{\CJKfamily{zhyahei}\sffamily\small}{}{}{}{}
\titlespacing{\subsection}{0pt}{4pt plus 2pt minus 2pt}{4pt plus 2pt minus 2pt}

% set page margins (assumes letter paper)
\geometry{body={7in, 9.0in},
	left=0.75in,
	top=0.75in}

% prevent paragraph indentation
\setlength\parindent{0em}

% set line spacing
\setstretch{1}

% define space between list items
\newcommand{\listitemspace}{0.25em}

% make unordered lists without bullets and use compact spacing
\renewenvironment{itemize}
{\begin{list}{}{\setlength{\leftmargin}{0em}
			\setlength{\parskip}{0em}
			\setlength{\itemsep}{\listitemspace}
			\setlength{\parsep}{\listitemspace}}}
	{\end{list}}

% make tabbed lists so content is left-aligned next to years
\TabPositions{\listtabwidth}
\newlist{tablist}{description}{3}
\setlist[tablist]{leftmargin=\listtabwidth,
	labelindent=0em,
	topsep=0em,
	partopsep=0em,
	itemsep=\listitemspace,
	parsep=\listitemspace,
	font=\normalfont}

% print only the month and year (Chiense) when using \today
\newdateformat{monthyeardate}{\THEYEAR 年 \THEMONTH 月}

% define hyperlink appearance and metadata for pdf properties
\hypersetup{
	colorlinks  = true,
	urlcolor    = Blue,
	citecolor   = blue,
	linkcolor   = red,
	pdfauthor   = {\myname},
	pdftitle    = {\myname: 简历},
	pdfsubject  = {简历}
}


\usepackage[multiple, hang, flushmargin, stable]{footmisc}
\usepackage{fnpct}
\addtolength{\skip\footins}{1.5pc plus 5pt}
\renewcommand\thefootnote{\textcolor{red}{\arabic{footnote}}}
\renewcommand\footnoterule{}

\usepackage{fancyhdr}
\usepackage{lastpage}

% Custom footer for the first page
\fancypagestyle{lastpagefooter}{
	\fancyhf{}
	\fancyfoot[L]{查看最新版本:\href{https://zixingjiang.com/cv/}{https://zixingjiang.com/cv/}}
	\fancyfoot[R]{第 \thepage 页, 共 \pageref{LastPage} 页}
	\renewcommand{\headrulewidth}{0pt}
	\renewcommand{\footrulewidth}{0.5pt}
}

% Custom footer for the rest of the pages
\pagestyle{fancy}
\fancyhf{}
	\fancyfoot[L]{查看最新版本:\href{https://zixingjiang.com/cv/}{https://zixingjiang.com/cv/}}
\fancyfoot[R]{第 \thepage 页, 共 \pageref{LastPage} 页}
\renewcommand{\headrulewidth}{0pt}
\renewcommand{\footrulewidth}{0.5pt}
\begin{document}
\raggedright{}
	
% display your name as the document title
\namefont{\myname} \hfill 更新于\monthyeardate\today
\vspace{0.5em}
\hrule width \hsize height 1pt \kern 1mm \hrule width \hsize 
% affiliation and contact info blocks
\vspace{1em}
\begin{minipage}[t]{0.600\textwidth}
	% current primary affiliation, left-aligned
	香港中文大学医学院\\
	外科学系在读硕士\\
	中国香港,新界,沙田区
\end{minipage}
\hfil
\begin{minipage}[t]{0.395\textwidth}
	% contact info details, right-aligned
	\flushright{}
	电邮:\href{mailto:zxjiang@surgery.cuhk.edu.hk}{zxjiang@surgery.cuhk.edu.hk} \\
	电话(香港):+852 5954 9660 \\
	个人主页:\href{https://zixingjiang.com}{https://zixingjiang.com}\\
\end{minipage}
	
\section*{教育背景}
\begin{tablist}
	\item[\textit{哲学硕士}]  \tab{}外科学,香港中文大学,在读,2024--\\
	\item[\textit{工学学士}]  \tab{}电子信息工程(计算机工程专修),\textit{甲等荣誉},香港中文大学(深圳),2023\\
\end{tablist}
	

\section*{工作经历}
	
\begin{tablist}
	\item[2023--24]   \tab{}香港中文大学外科学系 \hfill \textit{中国香港}\\
	研究助理,生物医学机器人实验室 \hfill \emph{\emph{2023} 年 \emph{11} 月至 \emph{2024} 年 \emph{7}  月}
		
	\item[2020--23]   \tab{}香港中文大学(深圳)机器人与人工智能实验室\hfill \textit{中国,广东,深圳}\\
	学生研究助理, 手术及协作机器人团队\hfill \emph{\emph{2023} 年 \emph{2} 月至 \emph{2023} 年 \emph{8}  月}\\
学生研究助理, 海洋机器人团队 \hfill \emph{\emph{2020} 年 \emph{9} 月至 \emph{2023} 年 \emph{2}  月}
\end{tablist}
	
\section*{研究领域}
\begin{itemize}
	\item 机器人学 / 医疗机器人 / 手术机器人
	\item 机器人图像引导介入
	\item 机器人辅助成像
\end{itemize}
	
\section*{部分项目经历}
\begin{tablist}
	\item[2023--24] \tab 机器人自主肺部超声检查 --- \emph{在香港中文大学\href{https://www.surgery.cuhk.edu.hk/profile.asp?alias=zli}{李峥教授}和\href{https://www.cse.cuhk.edu.hk/people/faculty/pheng-ann-heng/}{王平安教授}指导下参与的科研项目。该项目旨在使用机器人在重症监护室中自主为病人执行床边肺部超声检查,从而在疫情期间降低医护人员感染风险。我对该项目的贡献是作为机器人系统负责人,开发了一台机器人超声系统原型机并辅助对其的临床前测试。该原型机运行时的视频演示请见我的\href{https://www.zixingjiang.com/projects/robotic-lus/}{个人主页}\footnote{ \href{https://zixingjiang.com/projects/robotic-lus/}{https://zixingjiang.com/projects/robotic-lus/}}。}
	
	\item[2020--23] \tab 机械臂辅助无人机在受波浪扰动水面平台上降落 --- \textit{在香港中文大学(深圳)\href{https://sse.cuhk.edu.cn/en/faculty/qianhuihuan}{钱辉环教授}指导下参与的科研项目。该项目旨在使用机械臂辅助无人机在受波浪扰动的水面平台上降落。我对该项目的贡献是辅助博士研究生设计与验证机械臂的末端执行器和运动规划算法。详情请见我的\href{https://www.zixingjiang.com/projects\#marine-robotics}{个人主页}\footnote{ \href{https://zixingjiang.com/projects/floating-manipulator/}{https://zixingjiang.com/projects/floating-manipulator/} (机械臂运动规划算法),和 \\ \href{https://zixingjiang.com/projects/tethered-landing/}{https://zixingjiang.com/projects/tethered-landing/}  (机械臂末端执行器)}。}
\end{tablist}
	
\section*{发表著作\protect\footnote{\label{authorship}\NoCaseChange{用$^*$标记的为共同第一作者,用$^\dagger$标记的为通讯作者}}}
\subsection*{学术期刊论文}
\begin{tablist}
	\item[2024]  \tab{}R. Xu, \textbf{Z. Jiang}, B. Liu, Y. Wang, and H. Qian$^\dagger$, ``Confidence-Aware Object Capture for a Manipulator Subject to Floating-Base Disturbances,'' in \textit{IEEE Transactions on Robotics (T-RO)}, vol. 40, pp. 4396-4413, 2024,  doi: \href{https://doi.org/10.1109/TRO.2024.3463476}{10.1109/TRO.2024.3463476}.
\end{tablist}
	
\subsection*{学术会议论文}
\begin{tablist}
	\item[2023]   \tab{}Y. Jiang, R. Xu, \textbf{Z. Jiang} and H. Qian$^\dagger$, ``Design, Modeling and Control of A Novel USV-Manipulator System,'' \textit{2023 IEEE International Conference on Real-time Computing and Robotics (RCAR)}, Datong, China, 2023, pp. 206-211, doi: \href{https://doi.org/10.1109/RCAR58764.2023.10249802}{ 10.1109/RCAR58764.2023.10249802}.
		
	\item[2022]   \tab{}C. Liu, \textbf{Z. Jiang}, R. Xu, X. Ji, L. Zhang and H. Qian$^\dagger$, ``Design and Optimization of a Magnetic Catcher for UAV Landing on Disturbed Aquatic Surface Platforms,'' \textit{2022 International Conference on Robotics and Automation (ICRA)}, Philadelphia, PA, USA, 2022, pp. 1162-1168, doi: \href{https://doi.org/10.1109/ICRA46639.2022.9812270}{ 10.1109/ICRA46639.2022.9812270}.
\end{tablist}
	
\subsection*{专利}
\begin{tablist}		
	\item[2023]   \tab{}\textbf{蒋子星},冀晓强,刘崇锋,钱辉环,``一种四翼扑翼微型水面飞行器及飞行方法'',发明授权,公开号:\href{https://www.patentguru.com/cn/CN114889821B}{CN114889821B},公开日期:2023 年 2 月 24 日。
	
	\item[2022]   \tab{}冀晓强,宋泽方,\textbf{蒋子星},钱辉环,``一种扑翼机构及微型水面扑翼飞行器'',实用新型授权,公开号:\href{https://www.patentguru.com/cn/CN217320745U}{CN217320745U},公开日期:2022 年 8 月 30 日。 
	
	\item[2022]   \tab{}冀晓强,宋泽方,\textbf{蒋子星},钱辉环, ``一种基于双曲柄的扑翼机构及微型水面扑翼飞行器'', 实用新型授权,公开号: \href{https://www.patentguru.com/cn/CN217320744U}{CN217320744U},公开日期:2022 年 8 月 30 日。 
	
	\item[2022]   \tab{}刘崇锋,曹仲仲,\textbf{蒋子星},许若愚,冀晓强,钱辉环, ``无人机的降落系统,降落方法及储存介质'', 发明申请,实质审查,公开号: \href{https://www.patentguru.com/cn/CN115167522A}{CN115167522A},公开日期:2022 年 10 月 11 日。
\end{tablist}
	
	
	
\section*{会议活动}
\subsection*{研讨会报告\footref{authorship}}
\begin{tablist}
	\item[2024] \tab \textbf{Z. Jiang}, Y. Hu, X. Luo, J. Miao, Y. Zhang, L. Lei, S. Wang, P.-A. Heng, and Z. Li$^\dagger$, ``A Collaborative Robotic System with In-Plane Orientation Adjustment for Lung Ultrasonograph'', presented at workshop \textit{Autonomy in Robotic Surgery: State of the art, technical and regulatory challenges for clinical application}, 2024 IEEE International Conference on Robotics and Automation (ICRA), Yokohama, Japan, May 13, 2024. --- \emph{报告摘要,海报,及演示视频请见我的\href{https://zixingjiang.com/icra2024/}{个人主页} \footnote{\href{https://zixingjiang.com/icra2024/}{https://zixingjiang.com/icra2024/}}。}
\end{tablist}
	
\section*{学术服务}
\subsection*{同行评审}
\begin{itemize}
\item IEEE 机器人与自动化科学国际会议(\textit{The IEEE International Conference on Robotics and Automation}),2025
\item IEEE 机器人与仿生学国际会议(\textit{The IEEE International Conference on Robotics and Biomimetics}),2023 
\end{itemize}
	
	
\section*{领导力}
\begin{tablist}
	\item[2020--22]   \tab 香港中文大学(深圳)机器人与人工智能实验室下属学生机器人协会会长 --- \textit{职责:成员招募,协调社团活动,以及提供每周机器人社课教学。}
\end{tablist}
	
\section*{获得奖项}
\begin{tablist}
	\item[2023]   \tab 香港中文大学(深圳)理工学院 2022--23 学年院长嘉许名单
	\item[2021--22]   \tab 香港中文大学(深圳)第 17 至 19 轮本科生科研奖
\end{tablist}
	
\section*{专业技能}
\begin{tablist}
	\item[\textit{编程语言}] \tab Python,C++,C,MATLAB
	\item[\textit{机器人学}] \tab 全栈机器人开发经验,侧重运动规划和控制
	\item [\textit{图像处理}] \tab 时空域滤波, 图像分割, 图像配准
	\item[\textit{专业软件}] \tab 机器人开发: ROS,MoveIt,Gazebo,CoppeliaSim\\
	科学计算 / 数据分析 / 机器学习: Eigen,NumPy,pandas,PyTorch,scikit-learn\\
	二维及三维视觉:OpenCV,Open3D,3D Slicer\\
	计算机辅助设计:SolidWorks\\
	其他:Docker,DaVinci Resolve,\LaTeX
	\item[\textit{硬件设备}] \tab 开发平台:Linux,Arduino,树莓派,STM32,ESP32,FPGA\\
	机器人:机械臂,扑翼飞行器,无人机,无人车,无人船\\ 
	传感器:深度相机,六轴力传感器,光学定位跟踪系统\\
	交互设备: 触觉式力反馈设备,摇杆\\
	医学成像设备:医用超声
\end{tablist}


\section*{掌握语言}
\begin{tablist}
	\item[\textit{中文}] \tab 普通话 -- 母语
	\item[\textit{英语}] \tab 流利
\end{tablist}
\vfill
\hrule width \hsize \kern 1mm \hrule width \hsize height 1pt 
\section*{推荐人}
\begin{itemize}
	\item \textbf{李峥}~ 香港中文大学外科学系副教授\\
	关系: 硕士导师\\
	{\scriptsize \faEnvelope}\href{mailto:zhengli@cuhk.edu.hk}{zhengli@cuhk.edu.hk}

	\vspace{10pt}
	\item \textbf{钱辉环}~ 香港中文大学(深圳)理工学院副教授\\
	关系:本科毕业设计导师\\
	{\scriptsize \faEnvelope}\href{mailto:hhqian@cuhk.edu.cn}{hhqian@cuhk.edu.cn}\\
\end{itemize}
%\hrule width \hsize \kern 1mm \hrule width \hsize height 1pt 
% display today's date as Month Year after a vertical space below the end of the text
%\begin{center}
%	\vfill
%	更新于 \monthyeardate\today,查看最新版本:\href{https://zixingjiang.com/cv/}{https://zixingjiang.com/cv/}
%	\vspace{5pt}
%	\hrule
%\hrule
%\vspace{2pt}
%\hrule
%\end{center}

\end{document}